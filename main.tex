\documentclass{article}

\usepackage{hyperref}
\usepackage{comment}

\RequirePackage{fontspec}
\RequirePackage{xeCJK}

\setCJKmainfont{Droid Sans Fallback}
\XeTeXlinebreaklocale "zh"
\XeTeXlinebreakskip = 0pt plus 1pt

\begin{comment}
start: some settings for programming block
\end{comment}

\usepackage{listings}
\usepackage{color}

\definecolor{dkgreen}{rgb}{0,0.6,0}
\definecolor{gray}{rgb}{0.5,0.5,0.5}
\definecolor{mauve}{rgb}{0.58,0,0.82}

\lstset{frame=tb,
  language=Java,
  aboveskip=3mm,
  belowskip=3mm,
  showstringspaces=false,
  columns=flexible,
  basicstyle={\small\ttfamily},
  numbers=none,
  numberstyle=\tiny\color{gray},
  keywordstyle=\color{blue},
  commentstyle=\color{dkgreen},
  stringstyle=\color{mauve},
  breaklines=true,
  breakatwhitespace=true,
  tabsize=3
}

\begin{comment}
end: some settings for programming block
\end{comment}

\title{Today I learned.今日我所學}
\author{Nick Shek}
\date{2017-03-12}

\begin{document}

\maketitle

\clearpage

\tableofcontents

\clearpage

\section{MySQL}

\begin{itemize}
\item Setting up MySQL replication without the downtime: \href{https://plusbryan.com/mysql-replication-without-downtime}{https://plusbryan.com/mysql-replication-without-downtime}
\item mysql.sock文件作用(轉)

這個mysql.sock應該是mysql的主機和客戶機在同一主機上的時候,使用unix domain socket做為通訊協議的載體,它比tcp快通常遇到這個問題的原因就是你的mysql服務器沒運行起來。


Mysql有兩種連接方式:
(1),TCP / IP
(2),socket
對於mysql.sock來說明,其作用是程序與mysqlserver處於同一台機器,發起本地連接時可用。
例如你無須定義連接主機的具體IP得,只要為空或localhost就可以。
在這種情況下,即使你改變mysql的外部port也是一樣可能正常連接。
因為你在my.ini中或my.cnf中改變端口後,mysql.sock是隨每一次mysql server啟動生成的。已經根據你在更改完my.cnf後重啟mysql時間重新生成了一次,信息已跟著變更。

那麼對於外部連接,必須是要更改port才能連接的。

linux下安裝mysql連接的時候經常回提示說找不到mysql.sock文件,解決辦法很簡單:

如果是新安裝的mysql,提示找不到文件,就搜索下,指定正確的位置。

如果mysql.sock文件誤刪的話,就需要重啟mysql服務,如果重啟成功的話會在datadir目錄下面生成mysql.sock到時候指定即可。

如果還不行就選擇用TCP連接方式連接就行了,其實windows下還支持管道連接方式。

Source: \href{http://0001111.iteye.com/blog/1418638}{http://0001111.iteye.com/blog/1418638}
\end{itemize}

\section{Latex}

\begin{itemize}
\item Latex 中文化解決方案:
  \begin{itemize}
    \item \href{https://goo.gl/6ajvrM}{https://goo.gl/6ajvrM}
    \item \href{http://pangomi.blogspot.hk/2012/11/latex-in-ubuntu.html}{http://pangomi.blogspot.hk/2012/11/latex-in-ubuntu.html}
  \end{itemize}
\item Create Table of contents in Latex \href{https://www.sharelatex.com/learn/Table\_of\_contents}{https://www.sharelatex.com/learn/Table\_of\_contents}
\end{itemize}

\section{Prononciation}

\begin{itemize}
\item Jenkins: \href{https://www.youtube.com/watch?v=UqTD4gQz76g}{https://www.youtube.com/watch?v=UqTD4gQz76g}
\item Laravel: \href{https://www.youtube.com/watch?v=KHiRaK7hlCo}{https://www.youtube.com/watch?v=KHiRaK7hlCo}
\end{itemize}

\section{Linux}

\begin{itemize}
\item How to view files in binary in the terminal?

bin:
\begin{lstlisting}
xxd -b <file>
\end{lstlisting}

hex:
\begin{lstlisting}
xxd <file>
\end{lstlisting}

Source: \href{http://stackoverflow.com/questions/1765311/how-to-view-files-in-binary-in-the-terminal}{http://stackoverflow.com/questions/1765311/how-to-view-files-in-binary-in-the-terminal}

\end{itemize}

\section{Chrome}

\begin{itemize}
\item Create Your First Chrome App: \href{https://github.com/nickshek/latex-til}{https://github.com/nickshek/latex-til}
\end{itemize}

\end{document}
